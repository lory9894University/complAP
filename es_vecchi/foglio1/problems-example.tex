\documentclass{oxmathproblems}

\printanswers
\usepackage[italian]{babel}
\oxfordterm{CompAP}
\course{I Foglio Esercizi}
\sheetnumber{1}
\sheettitle{Gruppo Torrielli Federico - Sciandra Lorenzo}


\begin{document}

\begin{questions}

% Prima domanda
\miquestion
Prima risposta:
\begin{itemize}
    \item $E(X)$
    \item $Var(X)$
\end{itemize}

\begin{solution}
    \\
    Se scelgo A $\longrightarrow Poisson(2.6)$\\
    Se scelgo B $\longrightarrow Poisson(3)$\\
    Se scelgo C $\longrightarrow Poisson(3.4)$\\
    X='Numero di errori nel dattiloscritto'\\\\
    Y=$\begin{cases}
        1\text{ - se scelgo dattilografo A}\\
        2\text{ - se scelgo dattilografo B}\\
        3\text{ - se scelgo dattilografo C}
    \end{cases}$
    Tutti i casi hanno $P=1/3$\\\\
    Uso il teorema della doppia attesa:\\
    \begin{equation}
        EX=E[E[X|Y]=E[X|Y=1]\cdot1/3+E[X|Y=2]\cdot1/3+E[X|Y=3]\cdot1/3
    \end{equation}
    \[
       \begin{array}{l}
            E[X|Y=1]=2.6\\
            E[X|Y=2]=3\\
            E[X|Y=3]=3.4\\
        \end{array}
    \]
    Riprendendo l'equazione (1) e sostuendo:\\
    \[
    EX=1/3\cdot2.6+1/3\cdot3+1/3\cdot3.4=\frac{2.6+3+3.4}{3}=9/3=\textbf{3}
    \]\\
    Varianza: per calcolare la varianza di X $\longrightarrow VarX=EX^2-E^2X$\\
    $E^2X=(EX)^2=(3)^2=9$ già calcolato\\
    A questo punto calcoliamo $E[X^2]$\\
    $EX^2=E[E[X^2|Y]]=E[X^2|Y=1]\cdot1/3+E[X^2|Y=2]\cdot1/3+E[X^2|Y=3]\cdot1/3$\\
    Sapendo che il momento quadro di una Poisson è $\lambda^2+\lambda$ ottengo:\\
    \[
        \begin{array}{l}
           E[X^2|Y=1] = (2.6^2+2.6) \\
            E[X^2|Y=2] = (3^2+3)\\
            E[X^2|Y=3] = (3.4^2+3.4)
        \end{array}
    \]
    Sostituendo alla formula di prima otteniamo:\\
    $E[X^2]=(2.6^2+2.6)\cdot1/3+(3^2+3)\cdot1/3+(3.4^2+3.4)\cdot1/3=\frac{36.32}{3}\simeq12.11$\\
    Quindi, riprendendo la formula della varianza di X: $12.11-9=3.11$
\end{solution}

% Seconda domanda
\miquestion
Seconda risposta: ($\forall y > 0$)
\begin{itemize}
    \item $f_{X|Y}(x|y)$
    \item $P(X>2|Y=y)$
    \item $E(X|Y)$
    \item $E(X+Y^2|Y)$
\end{itemize}

\begin{solution}
    Conoscendo la probabilità congiunta di (X,Y) $f(x,y)$ definita solo per\\
    $x,y > 0$ tutti gli integrali nelle formule che seguiranno saranno calcolati da $0$ a $\infty$:
    \begin{itemize}
    \item La densità di probabilità di $X|Y$: $f_{X|Y}(x|y)=\frac{f_{x,y}(x,y)}{f_y(y)}=\frac{f_{x,y}(x,y)}{\int_0^\infty f_{x,y}(x,y) dx}=$\\$=\frac{ye^{-y(x+1)}}{\int_0^\infty ye^{-y(x+1)} dx}=\frac{ye^{-yx}e^{-y}}{e^{-y}}=ye^{-yx}$
    \item $P(X>2|Y=y)=\int_2^\infty f_{x|y}(x|y) dx=\int_2^\infty ye^{-yx}=e^{-2y}$
    \item Calcolo prima $E(X|Y=y)$, funzione di y, da cui otterrò direttamente la funzione della v.a. Y:\\ $E(X|Y=y)=\int_0^\infty x\cdot f_{X|Y}dx=\int_0^\infty x\cdot ye^{-yx}dx=\frac{1}{y}$\\
    Quindi: $E(X|Y)=\frac{1}{Y}$
    \item $E(X+Y^2|Y)=$ per linearità $=E[X|Y]+E[Y^2|Y]=E[X|Y]+Y^2=\frac{1}{Y}+Y^2=\frac{Y^3+1}{Y}$
\end{itemize}
\end{solution}

% Terza domanda
\miquestion
Terza risposta: 
\begin{itemize}
    \item $P(S_N=0)$
    \item $P(S_N=1)$
    \item $E(S_N|N=5)$
    \item $E(S_N|N)$
\end{itemize}

\begin{solution}
    \\
    $S_N = X_1 + X_2+...+X_N$ si configura come una distribuzione binomiale di N v.a. di Bernoulli.
    La difficoltà risiede quindi nella v.a. N di Poisson con parametro $\lambda$, dato che dovremmo agire considerando la binomiale per tutti gli n che N può assumere.\\
    \begin{itemize}
        \item $P(S_N=0)= \sum_{n} P(S\sb{N} = 0, N = n) = \sum_{n}  P(S\sb{n} = 0 | N = n)\cdot P (N = n) = \sum_{n} P (S\sb{n} = 0) \cdot P (N = n) = \sum_{n}\binom{n}{0}p^0(1-p)^{n-0}\cdot\frac{\lambda^n}{n!}e^{-\lambda}=\sum_{n}(1-p)^{n}\cdot\frac{\lambda^n}{n!}e^{-\lambda}$
        \item $P(S_N=1)= \text{ riprendendo i calcoli dal passo precedente =}\sum_{n}\binom{n}{1}p(1-p)^{n-1}\cdot\frac{\lambda^n}{n!}e^{-\lambda}=\sum_{n}np(1-p)^{n-1}\cdot\frac{\lambda^n}{n!}e^{-\lambda}$
        \item $E(S_N|N=5)=E(S_5|N=5)=$ per indipendenza $=E(S_5)=$ che è Binomiale(5,p) il cui valore atteso è $=5p$
        \item Partiamo col calcolare $E(S_N|N=n)$ possiamo riscriverlo come: \\
        $E(\sum_{i=1}^N X_i|N=n)=E(\sum_{i=1}^n X_i|N=n)$  = per linearità dell'attesa = \\ $=\sum_{i=1}^nE[X_i|N=n]=n\cdot(1P(X_i=1|N=n)+0P(X_i=0|N=n))=nP(X_i=1|N=n)=$ per indipendenza $=np$\\
        Da cui otteniamo $E(S_N|N)=Np$
    \end{itemize}
\end{solution}

% Quarta domanda
\miquestion
Quarta risposta:
\begin{itemize}
    \item $E(N)$
\end{itemize}

\begin{solution}
    Risolveremo il problema calcolando l'attesa della v.a. opportunamente condizionata. Procediamo come segue:\\
    Sia:
    \begin{itemize}
        \item N= numero di lanci necesssari per ottenere due 6 consecutivi con 1 dado.
        \item X= 
        $\begin{cases}
            \text{1 se al primo lancio abbiamo un 6}\\
            \text{2 altrimenti}
        \end{cases}$
    \end{itemize}
    Per il teorema della doppia attesa $E[N]=E[E[N|X]]=E[N|X=1]P(X=1)+E[N|X=2]P(X=2)=E[N|X=1]1/6+E[N|X=2]5/6$\\
    Analizziamo separatamente i due casi:
    \begin{enumerate}
        \item Primo caso: $E[N|X=2]=1+E[N]$ (Semplicemente si spreca un lancio)
        \item Secondo caso: $E[N|X=1]=1+(1\cdot1/6+(1+E[N])\cdot5/6)=\frac{12+5E[N]}{6}=2+\frac{5}{6}E[N]$
    \end{enumerate}
    Riguardo al secondo caso, il termine tra parentesi indica che oltre al primo lancio di cui sappiamo l'esito 6 si terminerà in un solo altro passo con probabilità 1/6, mentre con probabilità 5/6 avremo solamente sprecato un altro lancio.\\
    Sostituiamo dunque questi nella formula originaria, trovando:\\ 
    $E[N]=\frac{1}{6}\cdot (2+\frac{5}{6}E[N])+\frac{5}{6}\cdot(1 + E[N])\longrightarrow \frac{E[N]}{36}=\frac{7}{6} \longrightarrow E[N]=42$
\end{solution}

\end{questions}
\end{document}
